
In cheminfomatics the problem of fingerprint searching in a large database is well studied. In \citet*{aung2010indexing} they present a new index-based search method called ChemDex (Chemical fingerprint inDexing) for speeding up the fingerprint database search. They propose a novel chain scoring scheme to calculate the Tanimoto (Jaccard) scores of the fingerprints using an early-termination strategy. The fingerprints are horizontally sorted by the total number of 1's they contain. A vertical sorting or rearrangement then takes place such that most of the 1's get pushed towards the left. After the shuffling vertical splits/slices are created in the database. A two tier index structure is created based on total number of 1's each data point has plus the number of 1's in each split. Since we have vertical slices in the database, when a query compound is given they propose a slice by slice fragment filtering.The bounds are calculated in the first slice and only the compounds which cross the threshold are checked for in the second slice. Since they are laterally traversing the slices and filtering after each slice the number of candidates are decreasing at each step hence reducing the time as compared to when we process the fingerprints in full. They come with a scoring technique such that the partial score at each slice is the upper bound for the final similarity score.\\

This field has also been well motivated in \citet*{swamidass2007bounds}. As mentioned in the paper, fingerprint vectors are used to search large databases of small molecules, currently containing millions of entries, using various similarity measures, such as the Tanimoto or Tversky's measures and their variants. They derive simple bounds on these similarity measures and show how these bounds can be used to considerably reduce the subset of molecules that need to be searched.The paper proposes bounds for both single molecule as well as molecule molecule queries. The queries considered by the paper are based on threshold similarity cut-off with a query compound as well as top-k best set. The paper studied the speed-up achieved in query time against query size , query distribution, length of the chemical fingerprints, threshold cut-off for similarity and the total size of chemical database. They show through experiments that their approach achieves linear speed-ups in order of 1 or more magnitude for the fixed threshold query scenario while for top-k queries, they achieve sub-linear speed-ups in range of $O(D^{0.6})$ where D is size of database.\\

A different approach to the same approach can be seen in \citet*{nasr2010hashing} where to speed-up database searches, they proposed to add to each binary fingerprint a short signature integer vector of length M. For a given fingerprint, the i-component of the signature vector counts the number of 1-bits in the fingerprint that fall on components congruent to i modulo M. Given two signatures, they show how one can rapidly compute a bound on the Jaccard-Tanimoto similarity measure of the two corresponding fingerprints, using the intersection bound. These signatures allow one to significantly prune the search space by discarding molecules associated with unfavourable bounds. \\

