
%% A category with the (minimum) three required fields
%\category{H.4}{Information Systems Applications}{Miscellaneous}
%%A category including the fourth, optional field follows...
%\category{D.2.8}{Software Engineering}{Metrics}[complexity measures, performance measures]
%
%\terms{Theory}
%
%\keywords{ACM proceedings, \LaTeX, text tagging} % NOT required for Proceedings

\section{Introduction}
Chem-informatics is largely concerned with the task of mining large chemical datasets. Typical applications of this include drug discovery, catalyst analysis and experiment predictions. Usually chemicals are represented as "Chemical Fingerprints".Chemical fingerprints are used for identification of chemicals. They are characteristics or distinctive patterns which help describe them. The comparison of chemical molecules is difficult but if we convert the molecules into bit-strings or into vector format it makes it easier to do a comprehensive comparison. A primary reason for using chemical fingerprints is scalability. We know that sub-graph isomorphism is an NP-complete problem making it difficult to find similarity between two molecules represented in a graphical format. When the data is represented in a vector string, techniques are more scalable in terms of size of query molecules and time to accomplish a similarity detection task.\\

Chemical fingerprints can be built based on structure properties like substructure features. In binary fingerprints each bit represents the presence or absence of specific chemical property like substructure presence. For example the bits could represent the count of individual chemical atoms like carbon, oxygen, hydrogen or the number of saturated or unsaturated aromatic carbon only, nitrogen containing or non-aromatic rings. The reason why substructure presence is is an important feature is because it can be used to reduce or filter out candidate mismatches in the sub-graph isomorphism test process. So given two bit-strings or vectors for binary chemical  fingerprints we can intuitively say that the similarity of the two fingerprints is directly proportional to the number of shared bits. \\

	These fingerprints/feature vectors are designed by domain experts and typical features could include information like number of occurrences of particular substructures, presence of molecules, bonds between the molecules, etc. Generally the dataset is high dimensional and also highly sparse which inherently makes searching and indexing such chemical data sets challenging. \\
	
	For performing query operation on high dimensional data, it is generally observed that all the indexing techniques perform poorly, as compared to linear scan. This is mainly due to the curse of dimensionality . As dimensions increases the data points are scattered further away in space. In some techniques which use pivots to form clusters like the one we will use, it is difficult to form compact groups with a low cluster radius. Since we are unable to form compact representative groups well, more than often a range query requires a search through the entire database of chemical molecules with no chance of pruning occurring.  Hence, in our work we have used "Linear scan" as our base line technique and have come up with  methods to improve our performance. Our experiments will be evaluated against the linear scan technique. The results of the range search will be verified by comparing the range search result set against that obtained by a linear scan approach.\\

	Various scoring schemes like Tversky, Pearson, Dice, and Kulczynski are available for comparing two given fingerprints \cite{willett2006similarity,swamidass2007bounds}.The most widely used and accepted similarity measure for binary fingerprints is the Tanimoto similarity which is equivalent to the Jaccard Similarity index defined as the size of the intersection of the sets divided by the size of union of the sets. Here the similarity measure would correspond to the number of bits which are 1 for both the sets of the fingerprints divided by the number of bits for which atleast one of the fingerprints is 1. In mathematical terms,for binary fingerprints X and Y ,where $X_i$ is the $i^th$ bit of X we define Tanimoto similarity as\[ T_s(X,Y) = \frac{\sum \limits_i X_i \wedge Y_i}
{\sum \limits_i X_i \vee Y_i} \] Here we can show that $1-T_s$ is a proper distance metric preserving triangle inequality which will allow us to use index structures which exploit the property, like M-trees. One challenge which we face here is the extension of the said similarity measure /distance metric to non binary fingerprints. Non binary fingerprints are basically feature vectors where each feature value is a count and not necessary signifying absence or presence using bits \\

	Similarity between chemical molecules plays an important role in various aspects of biology and chemistry like protein ligand docking, biological activity prediction,reaction site modelling and mainly drug discovery. Chemical or molecular similarity  plays a pivotal role in modern techniques used to predict chemical compound properties, designing tailor-made chemicals with a predefined and desirable set of properties and most importantly in managing drug design studies by using large databases containing structures of available chemicals.\\

	The first technique we used is that of M-tree \cite{ciaccia1997indexing} based indexing. In this technique we have proposed a novel pivot selection technique, which results in a efficient search time for range queries. We follow this up by embedding the data in to Euclidean space using Lipschitz embedding. We then perform a detailed analysis of the data, and we show that this analysis naturally leads us to inverted indexing. The large size of the data and the sparsity of the data are the primary challenges which we have addressed in this work.\\

