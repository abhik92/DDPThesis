
\section{Conclusions and Future Directions}

In this work we were able to show the effectiveness of M-tree based approach on non-binary feature vectors. Though we could achieve up to 2-fold improvement in query time, we would want to improve this further by doing further analysis.We also were able to effectively exploit the sparsity in data to construct the inverted index, which had given up to 100-fold improvement in query time.The current work has opened up many more avenues for us to explore. We shall be looking into the following areas:\\
\begin{enumerate}
\item Use of streaming algorithm as mentioned in the previous section. 
\item We also observe that we can replace Binary Search Tree in the inverted index technique by a height balanced data structure to result in even more efficient index structure. 
\item One other direction which is a worthy endeavour is the exploitation of the fact that the maximum value taken by any feature is bounded. We may be able to come up with some good bounds on the distance metric based on this, these bound can be in turn used to provide good pruning.  
\item We will also be looking into other dimensionality reduction techniques like principal component analysis, fast map, etc. Since we are looking into exact range search queries, we must be able to give guarantees on the new mapping, preferably a contractive mapping which will be a challenge \\
\end{enumerate}
 